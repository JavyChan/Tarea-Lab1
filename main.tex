\documentclass{article}
\usepackage{graphicx} % Required for inserting images
\usepackage{amsfonts,amsmath,amssymb,amsthm} %AMS

\usepackage{hyperref}%Hyperlinks
\hypersetup{
    colorlinks = true,
    citecolor = blue
}
\usepackage{biblatex}%bibliografia
\addbibresource{bibi.bib} %bib file

\usepackage{enumitem}% customize ennumerate
\usepackage[spanish]{babel}%default to spanish gente
\usepackage{csquotes} %prettier quotes
\usepackage{lipsum} % lorem ispum generator

\usepackage{multicol} % more columns for writing
\setlength{\columnsep}{0.5cm} %column separation
\usepackage[margin=2cm]{geometry} %universal text separation


\title{Análisis experimental de la constante de elasticidad de un resorte de laboratorio, con métodos computacionales}
\author{
    \textbf{Autor: } Javier Chandía }
\date{9 de septiembre, 2024}
%%%%%%%%%%%%%%%%%%%%%%%%%%%%%
%Notas

% El mini-informe debe incluir:
% Título
% Resumen
% Introducción
% Teoría
% Procedimiento Experimental
% Resultados
% Discusión (de errores)
% Conclusiones 
% Apéndice con el código en Python utilizado.

%No hay requisito de bibliografía, nice

%Papers de ejemplo para el formato:

%statmech [https://arxiv.org/pdf/2408.15039]
%quantum** [https://arxiv.org/pdf/2409.03748]
%classical** [https://arxiv.org/pdf/2409.01203]
%libro?mathematicalphysics [https://arxiv.org/pdf/2409.03117]
%mathematicalphysics2 [https://arxiv.org/pdf/2409.02805]
%condensedmatter-mesonano**[https://arxiv.org/pdf/2409.03471]

%** usar esta de referencia

%documentacion multicol
%[https://ctan.dcc.uchile.cl/macros/latex/required/tools/multicol.pdf]
%%%%%%%%%%%%%%%%%%%%%%%%%%%%%%%%%%%%%%%

\begin{document}

\maketitle

\begin{multicols}{2}
\begin{abstract}
    Se realizaron mediciones para obtener la constante de elasticidad de un resorte de laboratorio, tras lo cual, mediante un gráfico, se observó su comportamiento lineal en un gráfico masa-desplazamiento, acorde a la teoría.
\end{abstract}
\begin{center}
    \section{Introduccion}
\end{center}
Las fuerzas y sus leyes, establecidas por Sir Isaac Newton, ya se han establecido como un pilar fundamental en la física, y ha sido corroborada incontables veces.

Los resortes son una de estas fuerzas, y mediante el análisis de esta fuerza en un sistema masa-resorte suspendido en gravedad, podemos obtener la constante de elasticidad $k$ despejando en términos de las otras variables. 

A partir de esto, podemos obtener un diseño experimenal para obtener esta constante que es dependiente del material del resorte.

\centering
    \section{Teoría}
\raggedright
Se conoce que \begin{align}
    \sum_{i} \vec{F}_i=\frac{d\vec{p}}{dt} \label{eq:1}
\end{align}
En un sistema en reposo de un resorte suspendido, las fuerzas que actúan solo son la gravedad y la fuerza elástica, donde:
\begin{align}
    \label{eq:2} \vec{F}_{k}&= - k y \hat{y}\\ 
    \vec{F}_{g}&= mg \hat{y} \label{eq:3}
\end{align}
donde $\hat{y}$ representa el vector unitario perpendicular al suelo, en dirección  hacia abajo, $y$ es la distancia desde nuestra elección de origen, $m$ es la masa del resorte y $g$ la constante de aceleración gravitacional.

Luego:
\begin{align*}
   \eqref{eq:2},\eqref{eq:3},\eqref{eq:1}\cdot\hat{y} \implies & -ky + mg = 0\\
   \implies & k=\frac{mg}{y}\\
   \implies & y(m) = \frac{g}{k} m \tag{$\star$}  \label{eq:estrella}
\end{align*}

\columnbreak

Si aumentamos la masa que sostiene el resorte, entonces aumentamos la distancia en la que se desplaza el resorte.
\begin{align}
    y(m+\Delta m) &= \frac{g}{k}(m+\Delta m)\label{eq:4}
\end{align}
Para distintas masas $\{\Delta m_i\}_{i\in \mathbb{N}}$, llamaremos a estos nuevos desplazamientos $y_i=\Delta y_i$, los cuales serán nuestros puntos de medición. 

\centering
\section{Procedimiento experimental}
\raggedright

\centering
    \subsection{Materiales}
\raggedright

\begin{enumerate}[start=1, label={\bfseries \roman*})]
\item 1 resorte resistente
\item 1 regla de aprox. 2 metros
\item Soportes para suspender el resorte y la regla
\item Pesas de $(0.25, 0.5, 1)[kg]$
\item Una balanza de laboratorio
\end{enumerate}

\centering
    \subsection{Montaje y registros}
\raggedright

\begin{enumerate}[start=1, label={\bfseries \arabic*})]
\item Se ensambla el sistema del resorte con sus soportes, de forma que quede perpendicular a la superficie del suelo.
\item Se ensamblan los soportes de la regla, de modo que sea posible registrar los desplazamientos del sistema masa-resorte.
\item Se mide la masa real de las diferentes pesas en la balanza
\item Se conectan las masas con el resorte, tales que midamos en intervalos equiespaciados de $\Delta m_i=0.25 [kg]$ los diferentes desplazamientos de la masa $y_i$
\end{enumerate}
\newpage

\centering
    \section{Resultados}
\raggedright

Se obtuvieron las siguientes mediciones:

\begin{enumerate}[start=1, label={\bfseries \roman*})]
\item Masas (en gramos) \\
Según los valores reales de las masas, obtenemos que:
\begin{align*}
    \left[\begin{array}{cc}
         \Delta m_1 &= 249.8 \pm 0.1  \\
        \Delta m_2 &= 499.4 \pm 0.1\\
        \Delta m_3 &= 749.2 \pm 0.2\\
        \Delta m_4 &= 999.2 \pm 0.1\\
        \Delta m_5 &= 1249.0 \pm 0.2\\
        \Delta m_6 &= 1486.6 \pm 0.2\\
        \Delta m_7 &= 1748.4 \pm 0.3
    \end{array}\right]
\end{align*}
\item Desplazamientos (en centímetros)\\
Obtuvimos las siguientes medidas para los desplazamientos
\begin{align*}
    \left[\begin{array}{cc}
        y_1 &= 8.8 \pm 0.1\\
         y_2 &=  16.2 \pm 0.05\\
         y_3 &= 24.1\pm 0.05 \\
         y_4 &= 32.4 \pm 0.05\\
         y_5 &= 40.5 \pm 0.3\\
         y_6 &= 48.2 \pm 0.05\\
         y_7 &= 56.0\pm 0.05
    \end{array}\right]
\end{align*}
\end{enumerate}
Luego, graficando los resultados en un gráfico desplazamiento/tiempo (ver figura \ref{fig:grafico_desp-masa}) obtenemos el siguiente ajuste de función:
\begin{align*}
    10^{-2}y_i &= \frac{9.81}{k}\cdot 10^{-3}\Delta m_i\\
    y_i &= 10^{-1}\frac{9.81}{k}\cdot \Delta m_i \tag{S.I}
\end{align*}
Para obtener $k$ mediante el gráfico del ajuste lineal, realizamos una busqueda binaria, de modo que se ajuste lo más cercano posible a todos los puntos, y tras 14 iteraciones, obtenemos que:
\begin{align*}
    \frac{g}{k}&\approx ((2^{-5}+2^{-11}+2^{-12}+2^{-13}+2^{-14}))\\
    \implies \frac{g}{k}&= 0.03216552734375\\
    \implies k &\approx 30.4984895636
\end{align*}
Puesto que $k$ fue encontrado por una búsqueda binaria, este valor de $k$ es solo un estimativo. Aún así, podemos asegurar que:
\begin{align*}
    k= 29.6\pm 1.75
\end{align*}
Pues el valor de $2^{-5}$ es la mínima cota inferior de las pendientes, y conlleva a un $k=31.39$. A medida que aumentamos la presición, el valor de $k$ va disminuyendo.

Similarmente, podemos ir acotando superiormente, y $10\frac{g}{k}=2^{-4}-2^{-6}-2^{-7}-2^{-8}$ es la máxima cota superior de las pendientes, y obtenemos que $k=27.904$.

Nota para colegas: Una alternativa a este método es obtener $k$ en cada medición, y realizar un análisis estadístico sobre $k$
\columnbreak

\centering
    \section{Discusión}
\raggedright

En el experimento, se obvservaron los siguientes tipos de errores:

\begin{enumerate}[start=1, label={\bfseries \arabic*})]
\item \emph{Errores de medición en la masa}\\
Al realizar las mediciones de las masas, obtenemos la incertidumbre de la medición ($0.1$). A medida que realizamos combinaciones de masas, sin medirlas, los errores se van propagando. Este llega a un máximo de $0.3$.

La forma de reducirlos es midiendo en la balanza todas las combinaciones de masas.
\item \emph{Errores en la medición del desplazamiento}\\
El error en el desplazamiento está dado por la incertidumbre en la medición ($0.05$). Los errores mayores están dados por errores humanos (de paralaje, por interferencia visual).

La forma de reducirlos es simplemente realizando un proceso más riguroso en un intervalo de tiempo mayor 
\item \emph{Errores en el cálculo de $k$}\\
El error en la estimación de $k$ está dado únicamente por el método utilizado, donde tomamos el menor y mayor valores posibles para $k$. Sin embargo, otros métodos pueden llevar a un valor más preciso para $k$, como despejar $k$ para cada valor de $i$ y realizar un análisis estadístico sobre todas las mediciones
\end{enumerate}

\centering
    \section{Conclusiones}
\raggedright

Se obtuvo una medida para la constante elástica del resorte de laboratorio, con $k\approx 30.49$ mediante el análisis del gráfico y la búsqueda binaria. 

Este método se puede generalizar para cualquier resorte, tras lo cual podemos realizar predicciones como la oscilacion periódica de un resorte a la que se le aplicó una fuerza externa en el tiempo, con la precisión dictada por la incertidumbre en la constante de elasticidad.

Finalmente, comprobamos la efectividad de un ajuste mediante busqueda binaria. Si bien no es perfecto, puede resultar una aproximación decente a los mínimos cuadrados, aunque pierde efectividad cuando hay más dispersión de datos.
\end{multicols}
\begin{figure}
    \centering
    \includegraphics[width=1\linewidth]{graficolab1.pdf}
    \caption{Gráfico desplazamiento/masa. Error en rojo, puntos en azul, ajuste en morado. El error es más pequeño que el punto.}
    \label{fig:grafico_desp-masa}
\end{figure}
\newpage
\section{Apéndice}
Descargar código fuente: \href{https://github.com/JavyChan/Tarea-Lab1}{Github/JavyChan}
\end{document}
